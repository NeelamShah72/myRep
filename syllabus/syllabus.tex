\documentclass[a4paper]{article}

\usepackage{geometry}
\usepackage{hyperref}
\usepackage{ltablex} 
\usepackage{multirow}
\usepackage{tabularx} 
\usepackage[table]{xcolor}
\usepackage{url} 

\newcolumntype{L}[1]{>{\hsize=#1\hsize\raggedright\arraybackslash}X}%

% Help taken from http://tex.stackexchange.com/ and https://www.sharelatex.com/

\renewcommand{\thesection}{\Roman{section}.} 
\setlength{\parindent}{0pt}
\setlength{\parskip}{5pt}

\begin{document}

\begin{center}

  {\LARGE\bf HABIB UNIVERSITY}\bigskip

  {\large\bf CS 101 Programming Fundamentals}\medskip

  {\it ``Whoa.''}\\--- Neo, The Matrix\medskip

  Fall 2017\medskip
  
  Class: TBD\\
  Labs: TBD

\begin{tabular}{lp{.7\textwidth}}
  Instructors:&  Shah Jamal Alam, Shahid Hussain, Umair Azfar Khan, Waqar Saleem \\
  Assistants:& Mirza Zaeem Baig, Aisha Batool, Neelam Shah, Ata ur Rehman\\
  Hours: & All contact information TBA over LMS\\
  Course Website:&  \href{https://lms.habib.edu.pk/}{LMS}\\
  Discussion Forum:& \href{https://habibedu.facebook.com/groups/1809008982742834/}{Workplace}\\
Course Prerequisites:&  None.\\
Content Area: & This course fulfils Formal Reasoning and is part of CS Foundation. It is required for a CS major, fulfills the requirements of a CS minor, and is a required computing course for EE majors. For other students, it can be counted as either a Free Elective, University Elective, SSE Elective, CS Elective, or CS requirement. 
\end{tabular}

\end{center}

\section{Rationale}

Most people today are users of various computing systems of diverse complexity, e.g. a smartphone, an ATM machine, a microwave control panel. As users, we provide instructions to computers through an interface which the computer has been {\it programmed} to run. The user is limited to the tasks that the interface or the currently running program allow.

In order to instruct the computer to perform new tasks, i.e. ones that it is not alreay programmed for, we have to write our own programs. The most fundamental way to communicate with a computer is in binary, using 0's and 1's. {\it High level languages} allow us to write programs in a syntax close to natural language 

This course teaches the fundamentals of computer programming using a high level language.

\section{Course Aims and Outcomes}

This course aims to develop expertise in the fundamental techniques used in computer programming. Through the process of writing programs to solve problems, the students will develop a problem solving approach and comfort with a high level language.

On successful completion of this course, a student will:
\begin{itemize}
\item correctly recall the applicable data type to store a given values
\item correctly write a loop
\item correctly write a conditional
\item correctly write and call a function
\item correctly perform recursive calls to a function
\item correctly perform input and output from console
\item correctly recall the use of the string data type
\item correctly recall the use of the list data type
\item correctly recall the use of the dictionary data type
\item given a moderately complex problem, use the above components to write a program in a high level language that correctly solves the problem
\item given compilation errors resulting from a computer program, correctly debug the program
\end{itemize}
This course will use python as the programming language.

\section{Format and Procedures}

There will be 2 lecture sessions of an hour each and a lab session of 3 hours every week. The lectures will introduce new concepts which will be practically implemented in the lab session. A short quiz will be conducted in most lab sessions on the material covered that week. Longer problems will be assigned in that week's homework. There will be new homework on most weeks. Unless otherwise indicated, the homework will be assigned on Friday evening and will have to be submitted by the morning of the next Friday.

On average, you are expected to devote 3 hours of work outside class for every hour of lecture.

The course content is divided into 3 modules and there will be an exam at the end of each module as per the following schedule.
\begin{tabular}{|l|l|}
  \hline
  Exam 1 & 10-13h, 23 Sep \\\hline
  Exam 2 & 10-13h, 28 Oct \\\hline
  Exam 3 & during finals week \\\hline
\end{tabular}

A schedule of lectures, labs, and course assessments is given at the end of this syllabus.

To facilitiate programming, submission and grading of programs, and course communication, we will make use of several online platforms in this course: \href{https://lms.habib.edu.pk}{LMS}, \href{https://pscs.habib.edu.pk/}{PeopleSoft}, \href{https://www.hackerrank.com/}{HackerRank}, \href{https://www.pythonanywhere.com}{Python Anywhere} and \href{https://habibedu.facebook.com/groups/1809008982742834/}{Workplace}. The use of each platform is as follows.
\begin{description}
\item [\href{https://lms.habib.edu.pk}{LMS}] for faculty to officially communicate course related information to you. This will automatically get forwarded to your email address.
\item [\href{https://pscs.habib.edu.pk/}{PeopleSoft}] for faculty to officially submit grades.
\item [\href{https://www.hackerrank.com/}{HackerRank}] for students to submit their labs, quizzes, and homework assignments.
\item[\href{https://www.pythonanywhere.com}{Python Anywhere}] to write, run, debug, and save python code using a browser.
\item [\href{https://habibedu.facebook.com/groups/1809008982742834/}{Workplace}] for discussion by the faculty and by the students on matters related to the course.
\end{description}

Following are some ground rules for the course.
\begin{description}
\item[Punctuality] Please respect deadlines. Submit your work by the indicated time. Incomplete work will receive partial credit. Late work will not be accepted or graded.
\item[Contesting marks] Concerns reagrding a score will be entertained by the head RA up to a week after its release. Concerns raised later will not be entertained.
\item[Grace marks] Requests for grace marks for whatever reason will not be entertained and each such request will result in a penalty of 1\% from the overall score.
\item[Behavior] You are expected to maintain a behavior befitting {\it Yohsin} and acknowledging the classroom as a place of learning, exploration, and experimentation. Please extend your TAs and RAs the same respect and consideration that you do to the faculty.
\end{description}

\section{Course Requirements}

%Whatever tasks and assignments you include in your course should be aligned with the specified learning outcomes (final learning state, skills, knowledge, attitudes and values the students leave the course with) you have defined and specified earlier.

The lectures in the classroom will closely follow the course text book. Assignemnts, labs, and quizzes will borrow heavily from the book. You are required to stay abreast of the book chapters as the course proceeds.

We will use several online platforms, listed above. It is your responsibility to become familair with their use. Many of them will send notifications to your email address. You are required to regularly check your email. Failure to check email or stay updted with the various platforms is not an acceptable excuse for missing course communication.

\noindent{\bf Required text}: {\it Think Python 2e}, Allen B. Downey. Available online at \url{http://greenteapress.com/wp/think-python-2e/}.

\section{Grading Procedures}
\label{sec:grade}

A single grade will be computed as follows to apply to both CS 101 and CS 101L.\\
\begin{tabular}{|l|l|}
\hline
Labs (n-1) & 10\% \\\hline
Homework (n-1) & 20\% \\\hline
Quizzes (n-2) &	20\% \\\hline
Exam 1 & 10\%\\\hline
Exam 2 & 15\%\\\hline
Exam 3 & 15\%\\\hline
Class Participation & 10\%\\\hline
\end{tabular}
\begin{tabular}{|l|l|l|}
  \hline
  \multicolumn{3}{|c|}{\bf GRADING SCALE}\\\hline
  LETTER GRADE & GPA POINTS & PERCENTAGE\\\hline
  A+ & 4.00 & [97, 100] \\\hline
  A & 4.00 & [93, 97) \\\hline
  A- & 3.67 & [90, 93) \\\hline
  B+ & 3.33 & [80, 90) \\\hline
  B & 3.00 & [75, 80) \\\hline
  B- & 2.67 & [70, 75) \\\hline
  C+ & 2.33 & [67, 70) \\\hline
  C & 2.00 & [63, 67) \\\hline
  C- & 1.67 & [60, 63) \\\hline
  F & 0.00 & [0, 60)\\\hline
\end{tabular}

The least scoring lab, the least scoring homework assignment, and the 2 least scoring quizzes will be omitted from the grade computation. Class participation is based on your conduct in class, your engagement in discussions in the classroom and on the online forum, and your regular submission of assigned work.

You will find that your grade increases significantly if you keep up with the book readings.\smallskip

%Keep in mind, as you decide the weighting for the different assignments and tasks you give students it will have a major impact on their effort distribution. For example, if you have many homework assignments and/or quizzes, but not any one of them will count significantly toward the final grade, students may invest less time and commitment to doing them. If a certain percentage of the students’ grades are based on class participation, what criteria will be used to make that assessment: quantity or quality? If quality, what determines quality?

\section{Attendance Policy}

Habib University requires that all freshmen and sophomores must maintain at least 85\% attendance and all juniors and seniors must maintain at least 75\% attendance for each class in which they are registered. Non-compliance with minimum attendance requirements will result in \underline{automatic failure} of the course and may require the student to repeat the course when next offered. This policy is at a minimum. Departments, schools, and individual faculty members \underline{may alter this policy to include stronger attendance requirements} and/or implement them for all levels of students.  It is the responsibility of the student to keep track of their own attendance and speak with their faculty member or the Office of the Registrar for any clarification.

{\bf In this course, you may miss up to \underline{6 lectures and 2 labs}.}

\section{Accommodations for students with disabilities}

In compliance with the Habib University policy and equal access laws, I am available to discuss appropriate academic accommodations that may be required for student with disabilities. Requests for academic accommodations are to be made during the first two weeks of the semester, except for unusual circumstances, so arrangements can be made. Students are encouraged to register with the Office of Academic Performance to verify their eligibility for appropriate accommodations.

\section{Inclusivity Statement}

We understand that our members represent a rich variety of backgrounds and perspectives. Habib University is committed to providing an atmosphere for learning that respects diversity. While working together to build this community we ask all members to:
\begin{itemize}
\item share their unique experiences, values and beliefs
\item be open to the views of others 
\item honor the uniqueness of their colleagues
\item appreciate the opportunity that we have to learn from each other in this community
\item value each other's opinions and communicate in a respectful manner
\item keep confidential discussions that the community has of a personal (or professional) nature 
\item use this opportunity together to discuss ways in which we can create an inclusive environment in this course and across the Habib community 
\end{itemize}

\section{Office hours}

Office hours will be scheduled and posted over LMS. During these hours the course staff will be available to answer questions and provide additional help.

\section{Academic Integrity}

Each student in this course is expected to abide by the Habib University Student Honor Code of Academic Integrity.  Any work submitted by a student in this course for academic credit will be the student's own work.

% For this course, collaboration is allowed in the following instances: \textbf{Assignments, Presentation, and Project}.

Scholastic dishonesty shall be considered a serious violation of these rules and regulations and is subject to strict disciplinary action as prescribed by Habib University regulations and policies. Scholastic dishonesty includes, but is not limited to, cheating on exams, plagiarism on assignments, and collusion.
\begin{description}

\item[PLAGIARISM:] Plagiarism is the act of taking the work created by another person or entity and presenting it as one's own for the purpose of personal gain or of obtaining academic credit. As per University policy, plagiarism includes the submission of or incorporation of the work of others without acknowledging its provenance or giving due credit according to established academic practices. This includes the submission of material that has been appropriated, bought, received as a gift, downloaded, or obtained by any other means. Students must not, unless they have been granted permission from all faculty members concerned, submit the same assignment or project for academic credit for different courses. 

\item[CHEATING:] The term cheating shall refer to the use of or obtaining of unauthorized information in order to obtain personal benefit or academic credit. 

\item[COLLUSION:] Collusion is the act of providing unauthorized assistance to one or more person or of not taking the appropriate precautions against doing so.
\end{description}

All violations of academic integrity will also be immediately reported to the Student Conduct Office.  

You are encouraged to study together and to discuss information and concepts covered in lecture and the sections with other students. You can give ``consulting'' help to or receive ``consulting'' help from such students. However, this permissible cooperation should never involve one student having possession of a copy of all or part of work done by someone else, in the form of an e-mail, an e-mail attachment file, a diskette, or a hard copy. 

Should copying occur, the student who copied work from another student and the student who gave material to be copied will both be in violation of the Student Code of Conduct. 

During examinations, you must do your own work. Talking or discussion is not permitted during the examinations, nor may you compare papers, copy from others, or collaborate in any way. Any collaborative behavior during the examinations will result in failure of the exam, and may lead to failure of the course and University disciplinary action.

Penalty for violation of this Code can also be extended to include failure of the course and University disciplinary action. 

\newpage
\section{Tentative Course Schedule}

The following may be adjusted in view of class needs as the semester progresses. Refer to \href{https://lms.habib.edu.pk/portal/site/4d8a039a-d54a-4c25-a8e7-1b55ff695a86/page/72961ceb-2733-402e-97fe-e48f87c22a9c}{the course wiki} for an up to date version.

\noindent\begin{tabularx}{\textwidth}{lp{.73\textwidth}}
  \hline
  \multicolumn{2}{l}{\bf Module 1. Programming Basics}\\\hline
  Week 2. 28 Aug--1 Sep & Chapter 1. The way of the program\\
  Week 3. 4 Sep--8 Sep & Chapter 2. Variables, expressions and statements\\
  & \underline{Note}: Eid$^*$: Sep 2--4; Add deadline: Sep 5\\
  Week 4. 11 Sep--15 Sep & Chapter 3. Functions\\
  Week 5. 18 Sep--22 Sep & Chapter 3. Functions\\
  & \underline{Bonus homework}: Chapter 4. Case study: interface design\\
  & \textcolor{red}{Exam 1: 23 Sep, 10--13h}\\\hline
  \multicolumn{2}{l}{\bf Module 2. Programming Structures}\\\hline
  Week 6. 25 Sep--29 Sep & Chapter 5. Conditionals and recursion\\
  & \underline{Note}: Drop deadline: Sep 25; Ashura: 29 Sep--1 Oct\\
  Week 7. 2 Oct--6 Oct & Chapter 5. Conditionals and recursion\\
  Week 8. 9 Oct--15 Oct & Chapter 5. Conditionals and recursion; Chapter 6. Fruitful functions\\
  Week 9. 16 Oct--22 Oct & Chapter 6. Fruitful functions\\
  Week 10. 23 Oct--29 Oct & Chapter 7. Iteration\\
  & \underline{Note}: Withdraw deadline: Oct 23\\
  & \textcolor{red}{Exam 2: 28 Oct, 10--13h}\\\hline
  \multicolumn{2}{l}{\bf Module 3. Data Structure Basics}\\\hline
  Week 11. 30 Oct--3 Nov & Chapter 8. Strings\\
  Week 12. 6 Nov--10 Nov & Chapter 8. Strings; Chapter 10. Lists\\
  & \underline{Bonus homework}: Chapter 9. Case study: word play\\
  & \underline{Note}: Iqbal day: Nov 9; Chehlum$^*$: Nov 10\\
  Week 13. 13 Nov--17 Nov & Chapter 10. Lists\\
  Week 14. 20 Nov--24 Nov & Chapter 10. Lists; Chapter 11. Dictionaries\\
  Week 15. 20 Nov--24 Nov & Chapter 11. Dictionaries \\ 
  Week 16. 4 Dec--8 Dec & Chapter 12. Tuples\\
  & \underline{Bonus homework}: Case study: data structure selection\\
  & \underline{Note}: Last class: 8 Dec\\
  Weeks 17,18. 11 Dec--22 Dec & \underline{Note}: Finals \\
  & \textcolor{red}{Exam 3: TBD}\\\hline
\end{tabularx}
$^*$ -- subject to sighting of the moon.
\end{document}
